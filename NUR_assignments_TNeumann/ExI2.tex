\section{Exercise 2: Vandemonde matrix}

The Vandermonde-matrix can be obtained to find an unique solution for a given  Langrangian polynomial .
The coefficients of the Langrangian polynomial are calculated as:

\begin{equation}
	y_i = \sum{j=0}{N-1} g_j x_i^j
\end{equation}

\subsection{2.a) Approximation with LU-decomposition}

The entries of the Vandermonde-matrix ($V_{ij} = x_i^j$) are given and can read in by the following script:
 
\lstinputlisting{vandermonde.py}

With the code the following result for \textbf{c} is obtained.

\lstinputlisting{2a_c_vdm.txt}

The distribution of the data points is vizualized in the plot above.

\subsection{2.b) Approximation with Nevilles algorithm}

The result of $P_{\lambda}(k)$ are given in:

\lstinputlisting{2b_neville_result.txt}

\subsection{2.d) Time previous sub-exercises}

Within \textit{timeit} the number of repititions considered to determine the runtime can be adjustested via the \textit{repeat}-option. The default is set to 1e7. In the script below the number of repetitions are choosen as xxx so that code runs completely within less than 1min.

\lstinputlisting{2d_timeit.py}
